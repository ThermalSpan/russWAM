\documentclass{article}
\usepackage{fullpage}

\begin{document}
\title{Independent Study: \\ Warren Abstract Machine}
\author{Russell Bentley}
\maketitle
\tableofcontents

\section{Project Overview}
The purpose of this project is to implement a Warren Abstract Machine (WAM.) This machine is the most common base for Prolog implementations. The WAM has a limited instruction set, and it is used by reducing Prolog programs to a series of these instructions that will then be executed by the WAM. This intermediate WAM code can be generated by other Prolog compilers, like GNU Prolog. At the end of the project I aim to have an efficient implementation of the Warren Abstract Machine that can run WAM code. In addition, I would like to put together a report detailing my process and significant problems or successes I had. Furthermore, this project is the first half of a two part project. Next semester I hope to to implement a Prolog parser and interpreter based on my own WAM implementation.

\section{Goals}
I have three broad goals for this project. These goals are a large part of how I feel that this independent study 
should be evaluated. 

My first goal is to gain more experience with Prolog, and logic programming in general. I would like to have a better understanding of how Prolog programs are constructed and used. In addition, I hope to learn techniques from implementing Prolog that would allow me to implement Prolog-like features in the future. 

The second goal is to gain more experience with writing larger software projects. At the end of the semester I should have a WAM implementation that is well modularized and easy for others to integrate into their projects. The code should be clean, well documented, and adhere to relevant best practices. The project will be developed using public tools and will be hosted on Github. While qualitative in nature, it is important to me that this project be developed professionally and to a high standard.

My last goal is to gain more experience with the language I write the project in. I hope to write the WAM in Haskell. However, I am flexible on this point if the project's adviser feels strongly that another language would be better suited. No matter the language of implementation, I will have to deal with a wide range of issues ranging from parsing input to using complex data structures. This will be a good opportunity to delve deep into a new language. 

\section{Resources}
There will be a wide range of online resources used for this project. However, there are two very important print resources I have identified.

This first is ``An Abstract Prolog Instruction Set'' by David H.D. Warren \cite{AICPub641:1983}. This paper was where WAM was originally detailed. 

The second is ``Warren's Abstract Machine: A Tutorial Reconstruction'' by Hassan A\"{\i}t-Kaci \cite{Ait-Kaci:1991:WAM:113900}. This paper will be extremely important as it was developed with close help from David Warren and represents the most detailed overview of the WAM available. 

In addition, GNU Prolog will serve as my reference implementation should I need to look at another WAM implementation. GNU Prolog will also serve to fill in the missing parts of my tool chain while I develop the WAM. 

\section{Final Grade}
The final grade for this project will be based on the software created and the final report. There are three considerations when grading these deliverables. The first thing to consider is the functionality and completion of the implementation. To get the complete grade I need a working WAM that can run WAM code. The second consideration would be the code quality and tool use. This means that the code should adhere to best practices, the commit log should be useful, and the implementation should be easy for someone else to build. The last consideration is the usefulness of the report. The report should be a comprehensive syou you neeyou need ummary of my process, allowing others to see the steps I took to create my implementation. In this way, the implementation and report will determine my final grade. 



\bibliographystyle{plain}
\bibliography{workscited}
\end{document}
